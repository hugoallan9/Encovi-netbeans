\pagestyle{soloarriba}
\clearpage

$\ $
\vspace{14.5cm}

\noindent\begin{tabular}{p{0.9cm}p{6.8cm}}
& 2016.$\,$ Guatemala, Centro América \\
&\Bold Instituto Nacional de Estadística\\[-0.4cm]
&\color{blue!50!black}\url{www.ine.gob.gt}\\[0.9cm]
\end{tabular}\\
\noindent\begin{tabular}{p{0.9cm}p{6.8cm}}
& Está permitida la reproducción parcial o total de los contenidos de este documento con la mención de la fuente. \\[0.5cm]
 
& Este documento fue elaborado empleando  {\Sans R}, Inkscape y {\Logos \XeLaTeX}.\\
\end{tabular} 


\clearpage



	
	\clearpage
	\newpage $\ $

$\ $
\vspace{0.0cm}

\begin{center}
	{\Bold \LARGE AUTORIDADES}\\[1cm]
	
	
	{\Bold \large \color{color1!89!black} JUNTA  DIRECTIVA} \\[0.4cm]
	
	{ \Bold Ministerio de Economía}          \\
	Titular: Rubén Estuardo Morales   \\
	Suplente: Edwin Giovanni Verbena De León   \\ [0.4cm]
	
	{\Bold Ministerio de Finanzas} \\
	Titular: Julio Héctor Estrada \\
	Suplente: Victor Manuel Alejandro Martínez Ruiz\\[0.4cm]
	
	{\Bold Ministerio de Agricultura, Ganadería y Alimentación} \\
	Titular: Mario Estuardo Méndez Cóbar   \\
	Suplente: Rosa Elvira Pacheco Mangandi \\ [0.4cm]
	
	{\Bold Ministerio de Energía y Minas}\\
	Titular: Luis Alfonso Chang Navarro  \\
	Suplente: César Roberto Velásquez Barrera \\ [0.4cm]
	{\Bold Secretaría de Planificación y Programación de la Presidencia}   \\
	Titular: Miguel Ángel Moir Mérida  \\
	Suplente:  Edna Abigail Alvarez Och\\ [0.4cm]
	{\Bold Banco de Guatemala} \\
	Titular: Julio Roberto Suárez Guerra \\
	Suplente: Sergio Francisco Recinos Rivera\\ [0.4cm]
	{\Bold Universidad de San Carlos de Guatemala de Guatemala} \\
	Titular: Murphy Olimpo Paiz Recinos   \\
	Suplente: Eduardo Antonio Velasquez Carrera   \\ [0.4cm]
	{\Bold Universidades Privadas} \\
	Titular: Miguel Ángel Franco de León \\             Suplente: Ariel Rivera Irías\\ [0.4cm]
	{\Bold Comité Coordinador de \ Asociaciones  Agrícolas, Comerciales,Industriales y Financieras} \\
	Titular: Juan Raúl Aguilar Kaehler \\
	Suplente:  Oscar Augusto Sequeira García  \\ [0.4cm]
	
	{\Bold \large \color{color1!89!black} GERENCIA}\\[0.2cm]
	Gerente:  Nestor Mauricio Guerra Morales        \\
	Subgerente Técnico:  Fredy Gómez Gómez\\
	Subgerente Administrativo Financiero:  Edwin Portillo Portillo\\
\end{center}
\clearpage

$\ $
\vspace{1cm}

\begin{center}
	{\Bold \LARGE EQUIPO RESPONSABLE}\\[2cm]
	
	{\Bold \large \color{color1!89!black} REVISIÓN GENERAL}\\[0.2cm]
	Carlos Mancia\\[0.8cm]
	
	
	{\Bold \large \color{color1!89!black} EQUIPO TÉCNICO}\\[0.2cm]
	Carlos Mancia\\
	Carlos Ortiz\\
	Nelson Santacruz \\
	Pamela Escobar\\
	Hugo García \\[0.8cm]
\end{center}


\cleardoublepage
$\ $\\[2cm]
\noindent {\Bold \huge Presentación}\\[2cm]
El Instituto Nacional de Estadística (INE) cumpliendo con su función de divulgar las estadísticas oficiales, presenta a las entidades públicas y privadas, universidades, centros de investigación, municipalidades, cooperación internacional, organizaciones no gubernamentales y a todas las entidades que tienen interés en conocer la realidad del país a nivel local, los resultados de la Encuesta de Condiciones de Vida 2014 para cada uno de los veintidós departamentos del país.\\

Con estos documentos se pretende contar con datos más focalizados sobre los diversos temas investigados en la Encovi, como la dinámica poblacional, las características de las viviendas y los hogares, la salud, la educación, el empleo y la incidencia de la pobreza y pobreza extrema. Se espera, además, que con la información aquí presentada se diseñen o redefinan políticas públicas con un mejor conocimiento de las condiciones de vida de los departamentos.\\

Estas publicaciones no hubieran sido posibles sin el apoyo brindado por los hogares que compartieron la información a los distintos encuestadores, razón por la cual el INE agradece profundamente a todos los hogares que abrieron sus puertas y colaboraron con la Encovi 2014.\\

El INE espera que esta información sea de utilidad para los distintos actores sociales y que con ella se implementen acciones basadas en evidencia que permitan, en el mediano plazo, que Guatemala logre alcanzar mayores niveles de desarrollo para todos los habitantes del país.

\cleardoublepage
